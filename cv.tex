\documentclass{mycv}

\name{Kang Liang}
\address{321 411, School of Earth and Space Science \\ University of Science and Technology of China, Hefei}
\email{lk340621@mail.ustc.edu.cn}
\homepage{http://home.ustc.edu.cn/~lk340621}
\github{kanglcn}
%\linkedin{xu-cheng}

\begin{document}

\maketitle%

\section{Research \\ Interests}

I am interested in InSAR processing, Geodynamics, Machine Learning. My current focuses include:

\begin{itemize}
  \item InSAR processing.
  \item Geodynamics.
  \item Machine Learning.
\end{itemize}

%\section{Professional \\ Experience}
%
%\subsection{Simon Fraser University}[Burnaby, BC, Canada]
%\begin{positions}
%  \entry{Visiting Scholar}{Mar 2020~--~Present}
%\end{positions}
%
%\begin{itemize}
%  \item Advisor: \href{https://www.cs.sfu.ca/~jpei/}{Prof.~Jian Pei}
%  \item Designing novel techniques to build future generation high performance blockchain systems.
%  \item Developing a prototype to demonstrate the effectiveness of the proposed system.
%\end{itemize}
%
%\subsection{Hong Kong Baptist University}[Hong Kong]
%\begin{positions}
%  \entry{Ph.D.\ Candidate}{Nov 2014~--~Feb 2019}
%  \entry{Senior Research Assistant / Post-doctoral Research Fellow}{Dec 2018~--~Present}
%\end{positions}
%
%\begin{itemize}
%  \item Advisor: \href{https://www.comp.hkbu.edu.hk/~xujl}{Prof.~Jianliang Xu}
%  \item Designed novel algorithms and indexes for cloud-based query services to support efficient verifiable query processing in a wide range of enterprise systems.
%  \item Developed novel techniques to enable integrity assured search in blockchain databases.
%  \item Resulted to several research papers published in top-tier journals and conferences.
%\end{itemize}
%
%\subsection{Syracuse University}[Syracuse, NY, USA]
%\begin{positions}
%  \entry{Visiting Scholar}{Sep 2017~--~Dec 2017}
%\end{positions}
%
%\begin{itemize}
%  \item Advisor: \href{https://tristartom.github.io}{Dr.~Yuzhe Tang}
%  \item Designed and implemented a memory-access pattern secure software system on Intel SGX\@.
%  \item Developed a dynamic program partitioning framework to support implementing a variety of external oblivious algorithms and achieving cache-miss obliviousness.
%\end{itemize}
%
%\subsection{Homebrew}[Hong Kong]
%\begin{positions}
%  \entry{Core Maintainer}{Feb 2015~--~Feb 2017}
%\end{positions}
%
%\begin{itemize}
%  \item \url{https://brew.sh}
%  \item Acted as one of the core maintainers for the open source project Homebrew, which is the most popular package manager on macOS\@.
%  \item Implemented several major features and improvements including better tap system, core/formulae split, sandbox system, portable Ruby, and many bug fixes.

\section{Education}

%\subsection{Hong Kong Baptist University}[Hong Kong]
%\vspace{-\parskip}%
%\begin{itemize}[label={}]
%  \item Ph.D.\ in Computer Science \printdate{Nov 2014~--~Feb 2019}
%  \item Dissertation: \href{https://repository.hkbu.edu.hk/etd_oa/620}{Authenticated Query Processing in the Cloud}
%  \item Advisor: \href{https://www.comp.hkbu.edu.hk/~xujl}{Prof.~Jianliang Xu}
%\end{itemize}

\subsection{University of Science and Technology of China}[Hefei, China]
\vspace{-\parskip}%
\begin{itemize}[label={}]
  \item Bachelor of Geophysics \printdate{Sep 2016~--~Jun 2020}
\end{itemize}

\section{Skills}

\begin{description}
  \item[Programming] C, Python,  Matlab, \LaTeX, Bash
  \item[Tools] Vim, Git, Linux
  \item[Languages] English, Mandarin
\end{description}

\section{Selected \\ Publications}%

%\publications{bachelorthesis.bib}
\section{Talks}

\begin{enumerate}
  \item
\end{enumerate}

\section{Awards}

\begin{itemize}
  \item Geoscience Climbing Scholarship, USTC \printdate{2019}
  \item National Encouragement Scholarship, USTC \printdate{2018}
  \item Physics Innovation Research Experimental Paper Competition Special Award, USTC \printdate{2018}
  \item 817 Alumni Awards Scholarship, USTC \printdate{2017}
  \item National Encouragement Scholarshi, USTC \printdate{2017}
\end{itemize}

\end{document}
