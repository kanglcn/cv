\documentclass{mycv}

\name{Kang Liang}
\address{Roy M. Huffington Department of Earth Sciences \\ Southern Methodist University, Dallas Texas 75275}
\email{kangl@smu.edu}
\homepage{https://kanglcn.github.io}
\github{kanglcn}
%\linkedin{xu-cheng}

\begin{document}

\maketitle%

\section{Research \texorpdfstring{\\}{i} Interests}

I am interested in promoting the accuracy, standardization and intelligence of Synthetic Aperature Radar Interferometry (InSAR), applying InSAR technique to landslides identification, mechanics understanding and modeling. My current focuses include:

\begin{itemize}
  \item Deep learning for accurate and intelligent InSAR analysis
  \item State-of-art data science practices, (e.g., massively parallel processing, cloud computing) for InSAR big data processing
  \item Landslides identification and modeling with multi-source observation
\end{itemize}

\section{Education}

\subsection{Southern Methodist University}[Dallas, Texas, U.S.]
\vspace{-\parskip}%
\begin{itemize}[label={}]
  \item Ph.D.\ in Geophysics \printdate{Aug 2020~--~now}
%  \item Dissertation: \href{https://repository.hkbu.edu.hk/etd_oa/620}{Authenticated Query Processing in the Cloud}
  \item Advisor: \href{https://www.smu.edu/Dedman/Academics/Departments/Earth-Sciences/People/Faculty/Lu}{Prof.~Zhong Lu}
\end{itemize}

\subsection{University of Science and Technology of China}[Hefei, China]
\vspace{-\parskip}%
\begin{itemize}[label={}]
  \item B.S. in Geophysics \printdate{Sep 2016~--~Jun 2020}
\end{itemize}

%\section{Skills}

%\begin{description} 
%  \item[Programming] C, CUDA, Python,  Matlab, \LaTeX, Bash 
%  \item[Tools] Vim, Git, Linux, Dask
%  \item[Languages] English, Mandarin
%\end{description}

\section{Publications}%

\publications{mypub.bib}

\section{Invited Talks}

\begin{enumerate}
	\item ``Offset tracking with geocoded SLC'', NASA OPERA Project Science Team \printdate{May 15th, 2024} 
\end{enumerate}

\section{Paticipated \\ Projects}

\subsection{\href{https://kanglcn.github.io/moraine/}{Moraine - Modern Radar Interferometry Environment; A simple, stupid InSAR postprocessing tool in big data era}}
\vspace{-\parskip}%
\begin{itemize}
  \item 5000+ downloads on conda-forge, 14000+ downloads on PyPI.
  \item Serve as author and maintainer.
  \item Advanced Persistant Scatterer/Distributed Scatterer processing with massively parallel computing support (multi-GPU/multi-core).
  \item Low latency, high resolution, interative data visualization.
\end{itemize}

\subsection{\href{https://github.com/nisar-solid/ATBD}{ATBD: Notebooks for NISAR Solid Earth Algorithm Theoretical Basis Document}}
\vspace{-\parskip}%
\begin{itemize}
  \item Implement jupyter notebooks for ATBD transient deformation requirement (663).
  \item Help revise the algorithm theoretical basis document.
\end{itemize}

\subsection{\href{https://github.com/OPERA-Cal-Val/calval-CSLC}{OPERA Coregistered Single Look Complex (CSLC) validation tools}}
\vspace{-\parskip}%
\begin{itemize}
  \item Help develop and validate jupyter notebooks for absolute and relative geolocation error for sentinel-1 SLC.
\end{itemize}

%\section{Awards}

%\begin{itemize}
%  \item Geoscience Climbing Scholarship, USTC \printdate{2019}
%  \item National Encouragement Scholarship, USTC \printdate{2018}
%  \item Physics Innovation Research Experimental Paper Competition Special Award, USTC \printdate{2018}
%  \item 817 Alumni Awards Scholarship, USTC \printdate{2017}
%  \item National Encouragement Scholarship, USTC \printdate{2017}
%\end{itemize}

\end{document}
